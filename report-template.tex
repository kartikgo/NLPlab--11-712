\documentclass[11pt,letterpaper]{article}
\usepackage{fullpage}
\usepackage[pdftex]{graphicx}
\usepackage{amsfonts,eucal,amsbsy,amsopn,amsmath}
\usepackage{url}
\usepackage[sort&compress]{natbib}
\usepackage{natbibspacing}
\usepackage{latexsym}
\usepackage{wasysym} 
\usepackage{rotating}
\usepackage{fancyhdr}
\DeclareMathOperator*{\argmax}{argmax}
\DeclareMathOperator*{\argmin}{argmin}
\usepackage{sectsty}
\usepackage[dvipsnames,usenames]{color}
\usepackage{multicol}
\definecolor{orange}{rgb}{1,0.5,0}
\usepackage{multirow}
\usepackage{sidecap}
\usepackage{caption}
\renewcommand{\captionfont}{\small}
\setlength{\oddsidemargin}{-0.04cm}
\setlength{\textwidth}{16.59cm}
\setlength{\topmargin}{-0.04cm}
\setlength{\headheight}{0in}
\setlength{\headsep}{0in}
\setlength{\textheight}{22.94cm}
\allsectionsfont{\normalsize}
\newcommand{\ignore}[1]{}
\newenvironment{enumeratesquish}{\begin{list}{\addtocounter{enumi}{1}\arabic{enumi}.}{\setlength{\itemsep}{-0.25em}\setlength{\leftmargin}{1em}\addtolength{\leftmargin}{\labelsep}}}{\end{list}}
\newenvironment{itemizesquish}{\begin{list}{\setcounter{enumi}{0}\labelitemi}{\setlength{\itemsep}{-0.25em}\setlength{\labelwidth}{0.5em}\setlength{\leftmargin}{\labelwidth}\addtolength{\leftmargin}{\labelsep}}}{\end{list}}

\bibpunct{(}{)}{;}{a}{,}{,}
\newcommand{\nascomment}[1]{\textcolor{blue}{\textbf{[#1 --NAS]}}}


\pagestyle{fancy}
\lhead{}
\chead{}
\rhead{}
\lfoot{}
\cfoot{\thepage~of \pageref{lastpage}}
\rfoot{}
\renewcommand{\headrulewidth}{0pt}
\renewcommand{\footrulewidth}{0pt}


\title{11-712:  NLP Lab Report}
\author{Kartik Goyal}
\date{April 26, 2013}

\begin{document}
\maketitle
\begin{abstract}
This work aims to build a morphological analyzer for the Spanish language using Finite State Transducers(FST). The common rules for inflections of various base forms are inspected and are encoded in the FST. This method is quite effective as FSTs are able to handle a lot of regular rules and adding rules or edition of the analyzer becomes very comvinient by using operators like composition of FST, union of FSTs etc. The analyzer is built using two 1000-type corpora as the development set. The analyzer is also run on a 10000 word corpus.
\end{abstract}


\section{Basic Information about Spanish}

Spanish is a Romance(Ibero romance group) language that originated in Castile region of Spain. Apart from the other Romance languages, it enriched it's vocabulary from Basque and Arabic. Spanish is closely related to other Iberian Languages like Italian and Portuguese with which, it has 82\% and 89\% lexical similarity respectively. Spanish is written in Latin script. The interrogative and exclamatory clauses are introduces with inverted question and exclamation marks. 

It is the second most spoken language by number of native speakers and is one of the 6 official languages of the united nations. It is spoken by around 329 million people all over the world. Spanish is the primary language of 20 countries worldwide. Apart from Spain, it is official language of many Latin American regions(Argentina, Chile, Colombia, Costa Rica, Cuba, Dominican Republic, Ecuador, El Salvador, Guatemala, Honduras, Mexico, Nicaragua, Panama, Uruguay, Venezuela, Peru,Puerto Rico). It is also, the official language of Equatorial Guinea in Africa.

There are some grammatical and lexical differences between the Spanish spoken in regions of Spain and the Spanish spoken in Latin America. The main grammatical variations between dialects of Spanish involve differing uses of second person pronouns. For eg. In most of Spain, 'ustedes' and 'vosotros' are used according to the degree of formality, but only 'ustedes' is used in Latin America. There are some important vocabulary differences too between the dialects.

Spanish has an S-V-O grammar and is a heavily inflected language. It has a two gender noun system and the inflections of nouns, adjectives and determiners are caused by the number and gender. There are about 50 conjugated forms per verb which are caused by tense, number, person, T-V distinctions(formal) for second person, mood, aspect and voice. The modifying constituents tend to be placed after their head words. Also, usually the adjectives are placed after nouns.

Spanish is a morphologically rich language and this tool aims to carry out a resonably accurate morphological analysis of the language. 

 

\section{Past Work on the Morphology of Spanish}

Tzoukerman and Liberman discus about a transducer which is essentially is a finite directed graph with edges labelled by relations between surface and lexical forms. 

Santiago Rodriguez and Jesus Carretero describe the formal approach behind their spanish morphological analyzer, COES. 

Atserias et al. discus about the morphosyntactic analyzer which can be executed in GATE environment.

Carmona et al. discus about MACO+, a tool for morphological analysis of spanish.

Sidorov et al. have developed a spanish morphological analyzer and have made their wordlist publicly available.

\section{Available Resources}

The reference grammar being used for this project has two sources:
\begin{itemize}
\item
`A Comprehensive Spanish Grammar'- Jacques de Bruyne
\item
`A Reference Grammar of Spanish'- R.E. Batchelor, Miguel Angel San Jose
\end{itemize}
The lexicon I am using is derived from the words generated by the morphological analyzer page at \url{http://www.cic.ipn.mx/~sidorov/agme/}.

The corpora to test the systemś performance, have been built from the Judicial weekly and its Gazette provided on the official site of 'The Supreme court of Justice of the nation' from mexico \url{http://www.scjn.gob.mx/Paginas/Inicio.aspx}. The file used for building the corpus was the 'September 2011' issue of the gazette.

These judicial proceedings contain clean data and are dominated by the formal official language. But, this data also contains quoted statements by authorities and people involved in the proceedings. Hence, some personal words are also expected in the corpus. All the numbers and words containing single letters like 'a' were removed as they are not interesting from the morphological point of view. 

A total of 23,000 types were found in the corpus. Corpus A and Corpus B have 1000 types each. The test corpus has 10,000 words.

This corpus is expected to be a fairly clean and diverse because apart from being an official record, it contains people's personal statements and opinions.


\section{Survey of Phenomena in Spanish}

Spanish is a morphologically rich language. The verbs, nouns, pronouns and adjectives demonstrate inflections.
\begin{itemize}
\item
{\bf VERBS:}

Handling the verb inflection is extremely important because they demonstrate extremely rich inflectional phenomena.
The verbs undergo inflection according to following categories:
\begin{itemize}
\item
Tense: Present, Imperfect, Preterite, Future, Perfect
\item
Number: Singular, Plural
\item
Person: First, Second, Third
\item
Mood: Indicative, Subjunctive, Imperative, Conditional
\item
Aspect: Perfective aspect, Imperfective aspect
\item
Voice: Active, Passive
\end{itemize}

The regular verbs can be classified into 3 categories based on their endings:
\begin{itemize}
\item	-ar ended- Most frequent
\item	-er ended- Fairly frequent
\item	-ir ended- Least Frequent
\end{itemize}
All the regular verbs are conjugated in a standard manner. All the verbs also havew gerund and past-participle forms
The perfective aspects involve inflection of the auxilliary verb `haber', which is attached to the past participle form of the root. These phenomena can be accounted for by focusing on the inflections of `haber' alone.
Irregular verbs have slightly different inflection rules than those of the regular verbs. Some of the irregular verbs are extremely frequent and imortant in Spanish namely 'Haber', 'Tener', 'Ser' and 'Estar'.
Other irregular verbs either change the stem's spelling or inflect in a different manner than the regular verbs. The radical changing verbs like 'pensar', 'holgar' change $e \rightarrow ie$ and $o \rightarrow ue$, when they inflect for present indicative and present subjunctive cases. Some verbs change $e \rightarrow i$, $o \rightarrow u$ etc. In some verbs, which end in -iar and -uar, the i or u is stressed in the present indicative and the present subjunctive forms. The diphthong in the stem of some infinitives becomes two syllables with the stress on the second syllable, when the stem of the verb is stressed. Many verbs ending in '-cer'(Nacer) and 'cir' inflect for prsent indicative singular form by ending in '-zco'.In a number of verbs(eg. Secar), spelling of stems ending in `c',`g',`gu',`qu' and `z' has to change before ending beginning with `e' in order for the pronunciation of the stem to be preserved.

\item 
{\bf NOUNS:}

The nouns inflect according to 2 genders X 2 Number cases. Nouns ending in -o are masculine, with the only notable exception of the word mano ("hand"); -a is typically feminine, with notable exceptions.
The nouns ending in `ción' are generally feminine.
A small set of words of Greek origin and ending in `-ma', `-pa', or `-ta' are masculine. Many nouns have same spellings for both female and male forms, thereby, making a deterministic conclusion about the gender of a noun difficult.

Plurals of nouns in Spanish  are formed in the following way:

	When the noun ends in an unstressed vowel, `-s' is added.

	When the noun ends in a consonant, a stressed vowel `-es' is added.

Nouns ending in `z', change the `z' to `c' before `es' in plural.

\item
{\bf ADJECTIVES:}

Adjectives too inflect according to 2 genders X 2 Number. The adjectives ending in `o' are male and their female forms end in `a'.

Adjectives ending in -ete,-ote are male and their female forms have the `e' replaced by `a'.

Adjectives ending in `-an', `-in' and `-on' are male and their female forms have a appended at the end.

The adjectives inflect with number in exactly the same way as nouns inflect with number.

The adjectives also inflect according to the degree of comparison.'-ísimo' at the end and `re-',`rete-', and `requte-' at the beginning are used for expressing a superlative quality.

Finally some adverbs are generated from adjectives by adding a suffix `-mente' to them. This is a very common phenomenon.

\item
{\bf ARTICLES:}

Both Definite and Indefinite articles inflect with 2 gender X 2 Number. They have a fixed form without any variations.

However `de' and `el' are combined to `del' and `a' and `el' are combined to `al'.

\item
{\bf PRONOUNS:}
They are seldom used but they too have fixed inflections with 2 gender X 3 Person. Interestingly, the clitic pronouns, which can be direct or indirect, generally follow the infinitive form of the verb(eg. `damelo'). 

\end{itemize}  
\section{Initial Design}
'FOMA' and 'lexc' were the tools used to build the morphological analyzer. 
This round of development focuses primarily on the nouns , adjectives and verbs. The lexicon, containing base forms, used is extracted from the list of words used by Sidorov et al. in their morphological analyzer. 
\begin{itemize}
\item
{\bf NOUNS:} The nouns are divided into these categories:
\begin{itemize}
\item Nouns ending in -o: These are masculine and inflect according to number. 
\item Nouns ending in -a: These are feminine and inflect according with number.
\item Nouns ending in -ción: These are feminine. Interestingly, their plural forms end in -en and the accent on ó is removed as it becomes the second last syllable which is emphasized by default in speech.
\item Nouns having other endings: They have -es plural endings and their gender cannot be determined unless we know of their origins.
\item A specialized list of nouns which have -ma, -pa, -ta endings was used as these nouns are masculine inspite of ending in a.
\end{itemize}
The plural forms of the nouns ending in -z, end in -ces instead of -zes.
Various rules were written to handle the irregular cases mentioned above.

\item
{\bf ADJECTIVES:} The adjectives are divided into these categories:
\begin{itemize}
\item Adjectives ending in -o: They were treated similarly to nouns, but an addtional case is handled. The adjectives ending -ísimo always denote the masculine superlative forms of the adjectives, which was handled by appropriate rules.
\item Adjectives ending in -a: Similar to previous case, adjectives ending in ísima are feminine superlative forms.
\item Adjectives Having other endings: These adjectives also were handled in a manner similar to the nouns. But, care was taken to handle the adverbs which are derived by inflecting the adjectives. Generally, -mente suffix is added to the feminine form of an adjective to convert it to an adverb.
\end{itemize}
The pluralization rules are exactly similar to the rules for nouns.

\item
{\bf VERBS:} The verbs are divided into these categories:
\begin{itemize}
\item Verbs ending in -ar
\item Verbs ending in -er
\item Verbs ending in -ir
\end{itemize}
All the regular rules for the verb inflections described in the previous section were implemented for all the categories of verbs. Among the irregular inflections, following are handled:
\begin{itemize}
\item `Haber', `Tener', `Estar' and `Ser' forms were hard-wired.
\item Verbs that change the stem from $e \rightarrow ie$, $o \rightarrow ue$, $i \rightarrow ie$, $u \rightarrow ue$, $e \rightarrow i$, $o \rightarrow u$  in the present indicative and subjunctive forms, were handled. 
\item The verbs ending in -cer and -cir, have a -zco ending for their present singular indicative form.
\end{itemize}

\end{itemize}

The articles and the pronouns were not hard-wired. Also, the clitics were not handled.
\section{System Analysis on Corpus A}
My informant, Jonathan Barker, manually evaluated the analyzer's performance on Corpus A. Since, guesses are also involved in the analyses for words not in the lexicon, the evaluation metric defined was simple and lenient. The recall of the system was calculated and guesses were evaluated to be correct if they managed to yield one or more correct analyses. The average number of guesses the system produces for a word is 5.

\vspace{5mm}

The recall on Corpus A with the above-mentioned metrics was {\bf 79.8 \%}. 


\section{Lessons Learned and Revised Design}
Based upon the feedback given by my informant, following shortcomings of the analyzer were listed:
\begin{itemize}
\item Clitics like `damelo´, which involve verbs following direct/indirect pronouns.  
\item Implementing basic infinitive forms of verbs.
\item Adjectival use of Past-Participles.
\item Verb inflections involving diphthongization and accent modification. 
\item Other irregular forms which rely on lexicons.
\item Closed forms groups like prononuns, articles, prepositions have not been implemented yet.
\end{itemize}
\section{System Analysis on Corpus B}

\section{Final Revisions}

\section{Future Work}





\bibliographystyle{plainnat}
\bibliography{refs}
\label{lastpage}
\end{document}
